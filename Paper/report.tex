\documentclass[10pt,twocolumn,letterpaper]{article}

\usepackage{cvpr}
\usepackage{times}
\usepackage{epsfig}
\usepackage{graphicx}
\usepackage{amsmath}
\usepackage{amssymb}

% Include other packages here, before hyperref.

% If you comment hyperref and then uncomment it, you should delete
% egpaper.aux before re-running latex.  (Or just hit 'q' on the first latex
% run, let it finish, and you should be clear).
%\usepackage[pagebackref=true,breaklinks=true,letterpaper=true,colorlinks,bookmarks=false]{hyperref}

\cvprfinalcopy % *** Uncomment this line for the final submission

\def\cvprPaperID{****} % *** Enter the CVPR Paper ID here
\def\httilde{\mbox{\tt\raisebox{-.5ex}{\symbol{126}}}}

% Pages are numbered in submission mode, and unnumbered in camera-ready
%\ifcvprfinal\pagestyle{empty}\fi
\setcounter{page}{1}
\begin{document}

%%%%%%%%% TITLE
\title{Autonomously Classifying Marine Life}

\author{Steven Hickson, Andrew Price \\
Georgia Institute of Technology\\
{\tt\small me@stevenhickson.com, arprice@gatech.edu}
}

\maketitle
%\thispagestyle{empty}


%%%%%%%%% BODY TEXT
\section{Introduction}
 Classification of elements in images has been a major topic in computer vision for a long time, due to its high degree of both difficulty and utility.
 One such application of object classification has been automated recognition of marine species.
 The ability to automatically identify species in an image would prove highly useful to both academic pursuits (eg. species surveying for a coral reef) and educational audiences (eg. a guided aquarium tour program).

\section{Related Work}
 Much work on classifying fish species has been done in highly controlled environments, usually requiring a carefully selected background and fish orientation \cite{white2006automated}.
 These approaches often require additional specialized equipment such as laser scanners or conveyor belts \cite{storbeck2001fish}, or require species dependent domain knowledge \cite{thonnat1988expert}.
  Instead, we intend to implement something more along the lines presented in \cite{berg2006animals}, which leverages a variety of features to perform a more general classification.
  However, we will not rely on text pulled from internet images for the classifier.

\section{Implementation}
We plan to use different types of feature detection that focus on shape, color, and texture in order to accurately classify marine life from a known data set.
We will compare and combine multiple feature detectors. These will include SURF \cite{bay2006surf} and ORB \cite{rublee2011orb} for texture, HoGs \cite{dalal2005histograms} for shape, and colored histogram similarity \cite{stricker1995similarity} for color. We will use a mixture of experts approach to create a SVM feature classifier with a higher level of object understanding. We will then compare a unified feature vector against a mixture of experts approach. Our system will have a priori knowledge of the environment and the different species classified. We can then find the best option for a classification system. We think that a combination of these features makes sense because marine life is so diverse. Some fish have a similar shape but vary in color or texture and vice versa. We hope that this unified feature vector will allow us to improve accuracy.

\section{Evaluation}
We plan to evaluate the classifier performance by generating a confusion matrix based on images drawn from a large subset of the species living at the Atlanta Aquarium. 
Additionally, we will compare the performance of the aggregated features against classifiers based on the individual features.
Finally, we will be comparing against \cite{berg2006animals} to determine the degree of dependency on text information.
We expect the concatenated features to outperform in situations where marine life shares a similar environment, and therefore shares similar color or shape characteristics.

%-------------------------------------------------------------------------

{\small
\bibliographystyle{ieee}
\bibliography{egbib}
}

\end{document}
